\documentclass[twoside,11pt]{article}

% Any additional packages needed should be included after jmlr2e.
% Note that jmlr2e.sty includes epsfig, amssymb, natbib and graphicx,
% and defines many common macros, such as 'proof' and 'example'.
%
% It also sets the bibliographystyle to plainnat; for more information on
% natbib citation styles, see the natbib documentation, a copy of which
% is archived at http://www.jmlr.org/format/natbib.pdf

\usepackage{jmlr2e}
%\usepackage{parskip}

% Definitions of handy macros can go here
\newcommand{\dataset}{{\cal D}}
\newcommand{\fracpartial}[2]{\frac{\partial #1}{\partial  #2}}
% Heading arguments are {volume}{year}{pages}{submitted}{published}{author-full-names}

% Short headings should be running head and authors last names
\ShortHeadings{95-845: MLHC Proposal}{Sroba}
\firstpageno{1}

\begin{document}

\title{Using Microsoft Kinect to Monitor Elderly Patients in the Home}

\author{\name Steve Sroba \email ssroba@andrew.cmu.edu \\
       \addr Heinz College\\
       Carnegie Mellon University\\
       Pittsburgh, PA, United States} 

\maketitle


\section{Proposal Details} \label{details}


\subsection{What is your proposed analysis? What are the likely outcomes?}

  The main objectives of this analysis are to improve on past efforts of classifying physical activity recorded by Microsoft Kinect, to find new ways to analyze performance on common clinical assessments of elderly patients such as gait, balance, and strength, and to predict type of participant performing the movement. I will attempt to identify markers for performance so that individual performance can be assessed over time, and compared with individuals of similar demographics. The likely outcomes are similar or improved accuracy for classification of different activities, and the creation of new metrics that can be extracted from Kinect activity data to assess health status of elderly patients.

\subsection{Why is your proposed analysis important?}

  The Kinect is	a 3D camera and microphone that can lead to change in health care such as increased virtual appointments, reduced transportation costs, service costs, and risk of infection. Daily assessments, physical therapy and exercise cam eliminate the need for frequent check-ups and office visits, and notify clinicians of declining health or risk of hospitalization. Games using Kinect can encourage active lifestyle or provide motivation for rehabilitation exercises\citep{cite1}.

  Around 1/3 of patients older than 65 and 1/2 older than 80 fall each year, and patients who have fallen, or have a gait or balance problem are at higher risk of falling again and losing their independence \citep{cite8}. On top of that, gait speed alone predicts functional decline and early mortality in older adults \citep{cite8}. For these reasons, assessing gait and balance may identify patients who need further evaluation.

  Performing physical assessments can help clinicians manage conditions typical of aging adults and prevent or delay complications. Functional status can be assessed at three levels: basic activities of daily living (BADLs), instrumental/intermediate activities of daily living (IADLs), and advanced activities of daily living (AADLs). Scales that measure functional status at each of these levels have been developed and validated. The Vulnerable Elders Scale-13 (VES-13) is a 13-item screening tool that is based upon age, self-rated health, and the ability to perform functional and physical activities. The Katz index for ADLs and the Lawton scale for IADLs are also common indices. Some AADLs can be ascertained by using standardized instruments, but AADLs are more complex, and less elders can complete them. The NIH toolbox is a collection of assessments, accompanied by a study of performance by different demographics for comparison of individual performance. Meta-analyses have found home assessments of this type to be effective in reducing functional decline as well as overall mortality \citep{cite8}. 
  
  Health care in the United States is extremely expensive for the quality of care that is given, and the elderly population accounts for a majority of health care spending. On top of that, the aging population is only going to make this crisis worse. Taking all of this into account, I believe this topic is extremely important analysis because it can lead to elimination of unnecessary office visits, prevention of adverse health events, and increased quality of care for patients, leading to higher reimbursements for physicians once quality-based reimbursement is in effect.


\subsection{How will your analysis contribute to existing work? Provide references.}

 	Home healthcare can help elderly live an independent lifestyle in their own home and avoid costs of specialist facilities.	Monitoring of normal activities, recognition of abnormal behavior, and detection of emergency situations can assure patient safety. Some examples that have been implemented are fall detection, as well as tremors and freezing in gait\citep{cite6}. Also, the assessment of posture and body movement can provide important information for screening and rehabilitation. Some examples that have been implemented are detection of reduction in range of motion after surgery, gait and movement assessment for risk of falling and to predict patients' ability to cope with daily practice after discharge from hospital, and presenting rehabilitation exercises as a game to motivate patients to perform otherwise repetitive exercises\citep{cite3,cite4,cite5}. 
 	
  \citet{cite6} had many similarities, however this analysis is attempting to improve upon it, and create new metrics to assess performance. For classification of the type of movement, new features will be engineered and new classification algorithms will be tested in an attempt to improve accuracy. For the goal of assessing individual performance, new metrics will be created by using the scales mentioned above so that individuals can be compared to those of similar demographics. The biggest difference is that this assessment will include an attempt to classify the type of participant performing the activity.

\subsection{Describe the data. Please also define Y outcome(s), U treatment, V covariates, W population as applicable.}

  The dataset was created by Manchester Metropolitan University. It is titled K3Da, A Kinect Healthcare Dataset: Benchmarking Healthcare Applications. It is a realistic clinically relevant human action dataset containing skeleton and depth data and associated participant information. Activities are based on the Short Physical Performance Battery\citep{cite2}, and were recorded in a lab-based indoor environment with a single Kinect One 3D sensor, fixed to a tripod and motions performed directly infront of the device. Raw skeleton files (text file) contain the user position and pose in real world space. Depth files are raw .bin files, with person index to identify the subject. Matlab skeletal files contain 25 predefined joints (X, Y, Z coordinates) defined by the Kinect SDK and the RGB mapped coordinates. It is noted that the depth data is very noisy, however it has been provided to aid in visual reference, and may be useful for imitating real-world data (not collected in a lab). The age, gender, weight, height and an indicator of the type of participant (1=young, 2=British Masters' Athlete and 3=old) was provided for each\citep{cite7}.

Population (W): 54 participants, 26 young (<=59 years of age), 14 elderly (>=61 years of age) and 14 British Masters Athletes.

Outcome (Y): One task is to classify the type of activity. The different activities are below:

  Balance (2-leg Open Eyes)
  Balance (2-leg Closed Eyes)
  Chair Rise
  Jump (Minimum)
  Jump (Maximum)
  One-Leg Balance (Closed Eyes)
  One-Leg Balance (Open Eyes)
  Semi-Tandem Balance
  Tandem Balance
  Walking (Towards the Kinect)
  Walking (Away from the Kinect)
  Timed-Up-and-Go
  Hopping
  
  Another is to classify the type of participant performing the activity. The possible outcomes are young, elderly, and elite athlete.
  
Treatment (U) and covariates (V) are not applicaple.

\subsection{What evaluation measures are appropriate for the analysis? Which measures will you use?}

 The evaluation measures that are appropriate (and that I will use) are accuracy (with confidence intervals), sensitivity, specificity, positive prediction value, negative prediciton value, ROC curves and PR curves.

\subsection{What study design, pre-processing, and machine learning methods do you intend to use? Justify that the analysis is of appropriate size for a course project.}

  The first step is to localize the skeleton in relation to the hip center of the first frame of the provided data with the code provided along with the data set\citep{cite7}. Data wrangling will then take place to get the data in the correct form to be analyzed. Signal processing such as filtering, variable sampling rates, and transformations to the frequency domain will follow, which overlaps with the next step: feature extraction/engineering. Again, Matlab code has been provided with the data set to visualize the data and profile joints as well as measure distance traveled. The x, y, and z coordinates of the joints over varying time windows will be the basis for most of the features. Identify repeating patterns in the activities will help engineer new features that have not been considered in the past.
  
  Next, feature selection/model tuning using k-fold ross validation will occur. Analysis using as many methods as time permits will be assessed. The current candidates are artificial neural networks, regression, k-Nearest Neighbor, Bayesian Network, Decision Tree, Support Vector Machine, and ensembles of these methods.
  
  To ensure that the project can be completed in time, I will go through each step, making sure to minimally complete each necessary step and not spend too much time/effort on one step. If time permits, I will then repeat and optimize the process of testing and finding the best models, and look into frameworks such as Microsoft Azure and TensorFlow.

\subsection{What are possible limitations of the study?}

  One limitation is that there are not enough subjects or data recored in order to accurate results.Another is the credibility of the creation of the data set. Lastly, the data collection environment may have been too controlled for the results to be applicable in the real world.

\bibliography{Proposal}
%\appendix
%\section*{Appendix A.}
%Some more details about those methods, so we can actually reproduce them.

\end{document}
